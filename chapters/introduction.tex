\section{Introduction} \label{sec:introduction}
With technology steadily improving over the years, people are becoming more interconnected than ever and entertainment in form of applications are plentiful. Despite this, depression continues to rise among the general population.
According to the World Health Organization, over 322 million cases of depressive disorder were recorded in 2017, representing a rise of around 18.4\% over the past ten years \cite{who_depression}.
Not only does depression negatively impact wellbeing, it may also lead to complications such as increased fatigue, decreased motivation or even suicide.
As a result, researchers have used computing technologies known as Affective Computing that seek to identify human emotions in order to combat mental disorders \cite{ieee_affective}.

In \cite{sohappy}, Moore, Galway and Donnelly propose a smartphone application using Affective Computing techniques to be used in a future study.
It is designed to encourage smiling, which has been shown to positively affect happiness and thus serves as an effective means to counteract depression.
The research objective of this report is to design and implement the aforementioned approach as an Android application.
In order to render the application suitable for different kinds of studies, the application's components are designed to be as interchangeable as possible, allowing individual adjustments to be made with ease.

This report summarises the work of students at the University of Applied 
Sciences Augsburg, as part of a student project supervised by Dr Leo Galway
from Ulster University, Belfast.

The report starts with an overview of appropriate approaches to implement the 
application in section \ref{sec:background}. In section \ref{sec:methodology}, 
the design choices and architecture of the app are described in detail, followed 
by a description of the implementation details in section \ref{sec:implementation}.

After presenting the development outcome and a discussion about the results in
section \ref{sec:results_and_discussion}, section \ref{sec:conclusion}
concludes the report with ideas for further work on the app.

