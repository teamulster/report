\section{Introduction} \label{sec:introduction}
With technology steadily improving over the years, people are becoming more interconnected than ever and entertainment in form of applications are plentiful. Despite this, depression continues to rise among the general population. According to the World Health Organization, over 322 million cases of depressive disorder were recorded in 2017, representing a rise of around 18.4\% over the past ten years \cite{who_depression}. Not only does depression negatively impact wellbeing, it may also lead to complications such as increased fatigue, decreased motivation or even suicide. As a result, researchers have used computing technologies known as \textit{Affective Computing} that seek to identify human emotions in order to combat mental disorders \cite{ieee_affective}.

In \cite{sohappy}, Moore, Galway and Donnelly propose a smartphone application to be used in a future study. It is designed to encourage smiling, which has been shown to positively affect happiness and thus serves as an effective means to counteract depression. The research objective of this report is to design and implement the approach described by Moore, Galway and Donnelly as an Android application. In order to render the application suitable for different kinds of studies, the application's components are designed to be as interchangeable as possible, allowing individual adjustments to be made with ease.

The report starts with an overview of appropriate approaches to implement the application in Section \ref{sec:background}. In Section \ref{sec:methodology}, the design choices and architecture of the app are described in detail, followed by a description of the implementation details in Section \ref{sec:implementation}. After presenting the development outcome in Section \ref{sec:results} and a discussion about alternative approaches in Section \ref{sec:discussion}, Section {sec:conclusion} concludes the report with ideas for further work on the app.

