\section{Conclusion} \label{sec:conclusion}

The current implementation of soHappy is developed for Android as target
platform. This limits the number of users able to use the app, as other 
mobile phones use different software. As an alternative approach, instead of
developing for Android, a platform independent approach would be benificial.
A platform independent app could be implemented using Flutter, a UI toolkit 
developed by Google being able to compile to all major platforms. Another
approach would be an based application, providing users a desktop as well as
mobile experience.

In spite of showing great potential, certain issues should be addressed in future
iterations of the soHappy app. Smile detection was implemented without the use of
facial landmarks, rendering it difficult for the app to distinguish between artificial
and genuine smiling. By using facial landmarks, analysis of smile detection results
would be greatly eased for researchers and a large amount of false positives
could be eliminated. Regarding user experience, users are recommended to use the app 
regularly and receive notifications whenever three hours have passed since they last
used the app. Such a timeframe may be less suitable for some users, making a
configurable notification setting more preferable. Additionally, during the core
process of the soHappy app, no feedback is provided to the user if their face can
no longer be detected. To remedy this, the already present color tint could be
changed to a different color if such a situation occurs.

