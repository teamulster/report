\chapter{Methodology} \label{sec:methodology}
% detailing the approaches that were used in your experiments; this may also include implementation details in relation to the mobile architecture used
According to \cite{sample_ref}, lorem ipsum dolor sit amet, consetetur sadipscing elitr, sed diam nonumy eirmod tempor invidunt ut labore et dolore magna aliquyam erat, sed diam voluptua. At vero eos et accusam et justo duo dolores et ea rebum. Stet clita kasd gubergren, no sea takimata sanctus est Lorem ipsum dolor sit amet. Lorem ipsum dolor sit amet, consetetur sadipscing elitr, sed diam nonumy eirmod tempor invidunt ut labore et dolore magna aliquyam erat, sed diam voluptua. At vero eos et accusam et justo duo dolores et ea rebum. Stet clita kasd gubergren, no sea takimata sanctus est Lorem ipsum dolor sit amet. Lorem ipsum dolor sit amet, consetetur sadipscing elitr, sed diam nonumy eirmod tempor invidunt ut labore et dolore magna aliquyam erat, sed diam voluptua. At vero eos et accusam et justo duo dolores et ea rebum. Stet clita kasd gubergren, no sea takimata sanctus est Lorem ipsum dolor sit amet.

\subsection{User Journey}
Lorem ipsum dolor sit amet, consetetur sadipscing elitr, sed diam nonumy eirmod tempor invidunt ut labore et dolore magna aliquyam erat, sed diam voluptua. At vero eos et accusam et justo duo dolores et ea rebum.

\subsection{Architecture}
Lorem ipsum dolor sit amet, consetetur sadipscing elitr, sed diam nonumy eirmod tempor invidunt ut labore et dolore magna aliquyam erat, sed diam voluptua. At vero eos et accusam et justo duo dolores et ea rebum.

\subsection{State Machine}
\usepackage{graphicx}

Figure \ref{fig:state_diagramm} shows the states and transitions of the state machine.


\begin{figure}
  \includegraphics[width=\linewidth]{../figures/state_diagramm.jpg}
  \caption{Figure 1: The state diagramm.}
  \label{fig:state_diagramm}
\end{figure}


A state chnage can be invoked throw three different kind of actions: An user action (pressing a button), a timeout after a specified couple of secondsm or an significant analyser result (faced detection or smile detection).
Note that if a face is detected and the user moves his face out of the frame afterwards, the process continues and handles this event indirectly as missing smile.
Consistently, the face leaving the camera frame after a first smile detection is not dealed with explicitly but leads to low smile results.
Besides there is no end knot, because like most smartphone applications the flow always returns to the start screen, represented in the start state.

\subsection{User Interface}
Lorem ipsum dolor sit amet, consetetur sadipscing elitr, sed diam nonumy eirmod tempor invidunt ut labore et dolore magna aliquyam erat, sed diam voluptua. At vero eos et accusam et justo duo dolores et ea rebum.

sss
