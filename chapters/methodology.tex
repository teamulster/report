\section{Methodology} \label{sec:methodology}
% detailing the approaches that were used in your experiments; this may also include implementation details in relation to the mobile architecture used
According to \cite{sample_ref}, lorem ipsum dolor sit amet, consetetur sadipscing elitr, sed diam nonumy eirmod tempor invidunt ut labore et dolore magna aliquyam erat, sed diam voluptua. At vero eos et accusam et justo duo dolores et ea rebum. Stet clita kasd gubergren, no sea takimata sanctus est Lorem ipsum dolor sit amet. Lorem ipsum dolor sit amet, consetetur sadipscing elitr, sed diam nonumy eirmod tempor invidunt ut labore et dolore magna aliquyam erat, sed diam voluptua. At vero eos et accusam et justo duo dolores et ea rebum. Stet clita kasd gubergren, no sea takimata sanctus est Lorem ipsum dolor sit amet. Lorem ipsum dolor sit amet, consetetur sadipscing elitr, sed diam nonumy eirmod tempor invidunt ut labore et dolore magna aliquyam erat, sed diam voluptua. At vero eos et accusam et justo duo dolores et ea rebum. Stet clita kasd gubergren, no sea takimata sanctus est Lorem ipsum dolor sit amet.

\subsection{User Journey}
Lorem ipsum dolor sit amet, consetetur sadipscing elitr, sed diam nonumy eirmod tempor invidunt ut labore et dolore magna aliquyam erat, sed diam voluptua. At vero eos et accusam et justo duo dolores et ea rebum.

\subsection{Architecture}
Lorem ipsum dolor sit amet, consetetur sadipscing elitr, sed diam nonumy eirmod tempor invidunt ut labore et dolore magna aliquyam erat, sed diam voluptua. At vero eos et accusam et justo duo dolores et ea rebum.

\subsection{State Machine}
Lorem ipsum dolor sit amet, consetetur sadipscing elitr, sed diam nonumy eirmod tempor invidunt ut labore et dolore magna aliquyam erat, sed diam voluptua. At vero eos et accusam et justo duo dolores et ea rebum.

\subsection{User Interface} \label{sec:user_interface}
Smartphone applications usually consist of multiple screens that the user can navigate through, each of them serving a different purpose. Such screens can be implemented in Android using the \texttt{Activity} and \texttt{Fragment} classes. An \texttt{Activity} object acts as an entry point to an Android app and provides a window for User Interface components to be created in \cite{intro_to_activities}. \texttt{Fragment} objects largely fulfill the same task, but are distinct from \texttt{Activity} objects in that they cannot persist on their own and must be hosted within an \texttt{Activity} object \cite{intro_to_fragments}. Since the soHappy app can only be started manually, a single \texttt{Activity} object for its sole entry point is sufficient. Each screen within the app is implemented with a \texttt{Fragment} object, which is hosted inside the single \texttt{Activity}. A small selection of \texttt{Fragment} objects are shown in figure. (TODO: Add figure here)

In terms of design, Android apps are generally expected to conform to Material Design, a set of guidelines defined by Google to help ensure both visual and practical consistency. As such, the user interface of the soHappy app is designed with Material Design in mind \cite{material_design}, beginning with the use of an \textit{App bar}, which contains the title of the current screen and thus helps the user in navigating around the app. Additionally, the app bar provides an overflow menu in the Home Fragment, allowing the user to navigate to various miscellaneous screens.

