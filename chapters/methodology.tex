\section{Methodology} \label{sec:methodology}
% detailing the approaches that were used in your experiments; this may also include implementation details in relation to the mobile architecture used
According to \cite{sample_ref}, lorem ipsum dolor sit amet, consetetur sadipscing elitr, sed diam nonumy eirmod tempor invidunt ut labore et dolore magna aliquyam erat, sed diam voluptua. At vero eos et accusam et justo duo dolores et ea rebum. Stet clita kasd gubergren, no sea takimata sanctus est Lorem ipsum dolor sit amet. Lorem ipsum dolor sit amet, consetetur sadipscing elitr, sed diam nonumy eirmod tempor invidunt ut labore et dolore magna aliquyam erat, sed diam voluptua. At vero eos et accusam et justo duo dolores et ea rebum. Stet clita kasd gubergren, no sea takimata sanctus est Lorem ipsum dolor sit amet. Lorem ipsum dolor sit amet, consetetur sadipscing elitr, sed diam nonumy eirmod tempor invidunt ut labore et dolore magna aliquyam erat, sed diam voluptua. At vero eos et accusam et justo duo dolores et ea rebum. Stet clita kasd gubergren, no sea takimata sanctus est Lorem ipsum dolor sit amet.

\subsection{User Journey}
Lorem ipsum dolor sit amet, consetetur sadipscing elitr, sed diam nonumy eirmod tempor invidunt ut labore et dolore magna aliquyam erat, sed diam voluptua. At vero eos et accusam et justo duo dolores et ea rebum.

\subsection{Architecture}
Lorem ipsum dolor sit amet, consetetur sadipscing elitr, sed diam nonumy eirmod tempor invidunt ut labore et dolore magna aliquyam erat, sed diam voluptua. At vero eos et accusam et justo duo dolores et ea rebum.

\subsection{State Machine}
Lorem ipsum dolor sit amet, consetetur sadipscing elitr, sed diam nonumy eirmod tempor invidunt ut labore et dolore magna aliquyam erat, sed diam voluptua. At vero eos et accusam et justo duo dolores et ea rebum.

\subsection{User Interface}
Lorem ipsum dolor sit amet, consetetur sadipscing elitr, sed diam nonumy eirmod tempor invidunt ut labore et dolore magna aliquyam erat, sed diam voluptua. At vero eos et accusam et justo duo dolores et ea rebum.

\subsection{Model Training}

As stated in the background section, soHappy uses a deep learning model to 
provide information about a person smiling by utilizing TFLite.

However, for TFLite to detect smiling, the model must be trained first. This
model is not trained on the device, but will be trained in before and shipped
with the android app.

The soHappy smile detection model is on based a convolutional neural network 
(CNN). CNNs are useful for image analysis purposes. A CNN utilizes multiple
convolutional layers as well as pooling layers. A convolutional layer works by
applying multiple different filters onto the input of the layer. An 
convolutional layer also looks at regions of the applied filter layers instead 
of individual pixels. This process is called convolution.
Pooling describes the process of discarding unused information.
Refer to \cite{8308186} for a more in-depth explanation on how CNNs work.

For the soHappy smile detection model, eight convolutional layers are used. 
Every two convolutional consecutive layers are followed by a pooling layer, 
repeated four times. In the end, a model is flattened. This model
architecture is based on the work of Madnani \cite{MayurMadnani2018}.

In order to train the model, the model of \cite{Arigbabu2015} was utilized.