\section{Background} \label{sec:background}
% identifying existing approaches within the research literature; this may potentially include mobile-based and non-mobile approaches if necessary
The renaissance of machine learning in the 2010s outclassed this approach because it lacks the ability to detect emotions/smiles.
Regarding soHappy, it led to TensorFlow becoming a more considerable option for emotion detection.
TensorFlow is a open source machine learning library developed by Google.
It supports a wide variety of programming languages, mainly targeting web and mobile applications.
For mobile development TensorFlow provides a Lite version called TensorFlow Lite (TFLite) which can be easily included in a mobile app.
The main advantage TFLite offers, is the ability to add a self-trained model to detect smiles. This is very helpful to reduce its more advanced emotion detection features to a simpler smile detection model.
As a second option we added a Face/Smile Detector implemented in Google's Android API (ML Kit) which also uses TensorFlow.
ML Kit comes with a model trained using deep machine learning (DML).
DML increases both face and smile detection performance by a big margin.
Therefore it has been added as a second option besides the OpenCV/self-trained TFLite implementations.
