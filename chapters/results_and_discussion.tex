\section{Results and discussion} \label{sec:results_and_discussion}
% some sort of visualisation of the results from research and/or results from the implementation of the mobile app
Lorem ipsum dolor sit amet.


Regarding the soHappy smile detection machine learning model, the following
result is present: After the training is finished, it is evaluated with the
test data. As Figure \ref{fig:training_result} shows, the accuracy detecting a
smile as a smile is about 98\%, with an overall evaluation accuracy of about 
94.5\%.

\begin{figure}
  \includegraphics[width=\linewidth]{figures/training_result.png}
  \caption{Heatmap describing percentages of true-positives and true-negatives as well as false-negatives and false-positives.}
  \label{fig:training_result}
\end{figure}

While those numbers look promising, using this model in production shows
room for improvement. Sometimes it is not enough to smile with a closed
mouth. Instead, an open mouth is required.

In the future, the soHappy model could be improved by improving the machine 
learning architecture. Also, a bigger dataset including more photographs of
people smiling with their mouth closed could significantly increase 
performance.